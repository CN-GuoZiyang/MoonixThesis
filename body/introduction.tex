% !Mode:: "TeX:UTF-8"

\chapter{绪论}[Introduction]

\section{课题来源及研究的目的和意义}

指令集架构(ISA),是计算机的一种抽象模型,是计算机体系结构模型中与程序设计相关的部分。通常,指令集架构定义了一系列支持的数据类型、寄存器、内存管理、基本特性与IO模型。计算机处理器的指令集架构有以下常见的两种:复杂指令集架构(CISC)、精简指令集架构(RISC),前者的代表就是x86架构,而后者的代表就是ARM架构。无论如何划分,不可否认的是,当今最流行的几个指令集架构,如x86和ARM,都是商业闭源的,在使用这些指令集架构时,需要向Intel或ARM控股缴纳高额的许可费。但是,Krste Asanović等人提出了这样一个观点:指令集架构应当是开放的,并以RISC-V作为开源指令集架构的典型进行论述。

RISC-V是一个基于精简指令集原则的开源指令集架构,该项目2010年始于加州大学伯克利分校,但是该项目的贡献者更多的是校外的志愿者和行业工作者。尽管不是第一个开源指令集,但是由于RISC-V适用于各种设备,譬如云计算机、PC机、移动设备和嵌入式系统,且其开源免费的特性使得它的社区十分活跃。于是RISC-V架构一经出现,就有许多芯片设计者以此为架构设计SoC。

相较于以功耗低著称的ARM架构处理器,RISC-V性能更强、面积更小、功耗更低,且相对于ARM,指令更加丰富,可扩展性更强,并且开源免费。或许是感受到了来自RISC-V的巨大压力,ARM控股甚至上线了一个网站来专门质疑RISC-V指令集。而相对于Intel的x86指令集架构,RISC-V没有那么多的历史包袱,不需要考虑向后兼容问题,可以说是“轻装上阵”,在设计操作系统时就无须考虑类似“A20地址线”的问题。而在关键技术被其他国家卡脖子的当今中国,开源免费的RISC-V可能有助于我国打破芯片壁垒。

同时,随着我国逐渐重视核心技术国有化,保障我国核心产业自主可控,我国开始大力发展自研芯片和自研操作系统。然而,相关工作大都是在企业和研究院层面展开,并没有下沉到高校层面的,形成了产学断层。这使得对于RISC-V相关技术的研究和教学,如RISC-V架构的芯片和支持RISC-V的操作系统的相关方向,成为国内高校研学的一片空白地带。

目前,大部分高校的操作系统的课程,都是基于x86架构设计的。然而,x86架构向后兼容的特性,使得即使是较新的x86操作系统,也不得不兼容一些过时的、已经被废弃不用的特性,例如上文提到的A20地址线。这使得学生对于操作系统本身的概念,会和x86一些特性混淆。令学生可能会更专注于繁琐而多余的x86特性,而不是聚焦于操作系统领域的通用概念。而RISC-V架构的操作系统,由于架构本身的精巧轻盈,更能让学生着眼于系统自身的设计。

基于以上原因,本文讨论了一种基于C语言和RISC-V架构的操作系统的设计方案,并给出其实现代码,旨在推动RISC-V向高校课程和实验室的下沉,以期能够反哺开源社区,最终客观上促进社区发展和RISC-V在国内的传播。

\section{国内外研究现状及分析}

\section{研究内容与方案}