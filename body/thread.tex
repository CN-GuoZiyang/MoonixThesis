% !Mode:: "TeX:UTF-8"

\chapter{线程调度模块设计与实现}[thread]
\label{chapter:thread}

\section{模块概述}

进程是一个处于运行状态的程序实例,拥有自己的内存空间和运行时数据。Moonix采用了一个最简单的进程模型:进程和线程一一对应,即一个进程中只包含一个线程。这样,CPU实际上只是在调度线程,进程剥离出线程之后,就只剩下资源分配的属性了。Moonix中,一个进程保有一个页表和一个文件描述符数组:页表代表这个进程的独立的虚拟地址空间,文件描述符则表示这个进程打开的文件。

当一个线程运行时,能代表一个线程的运行状态的切面数据,称为这个线程的上下文。这个切面由CPU中所有的寄存器和栈中的数据组成。理论上,只要恢复线程上下文,就可以恢复线程保存上下文时的状态。和中断类似,在线程切换时,需要将上一个线程的上下文保存起来,以便在未来某个时刻重新切换回这个线程时,能够继续之前的状态运行下去。内核中线程的数据结构中保存了该线程的上下文地址,由于上下文总是存放在线程栈顶,所以该地址也是栈顶地址。

线程调度依赖于时钟中断。当时钟中断发生时,内核采用Round-Robin算法\cite{DBLP:journals/eor/RasmussenT08},检查当前线程的时间片,并决定是否将暂停当前线程的运行,而将CPU资源让出。

Moonix中线程状态由线程池维护,线程的状态分为四种,分别如下:

\begin{itemize}
	\item 就绪状态(Ready):线程处于就绪队列中,等待被 CPU 调度执行
	\item 运行状态(Running):线程此时正占有 CPU 运行
	\item 睡眠状态(Sleeping):线程等待某个条件满足,此时线程不会被调度执行,直到条件满足后才会被加入就绪队列
	\item 退出状态(Exited):线程已经完成了任务执行,但是还没有回收资源
\end{itemize}

状态之间的切换及条件如图 \ref{pic:threadstatus} 所示。

\begin{figure}[htpb]
	\centering
	\includegraphics[width=0.7\textwidth]{threadstatus.png}
	\setlength{\abovecaptionskip}{2pt}
	\caption{保存中断上下文}
	\label{pic:threadstatus}
\end{figure}

\section{线程的启动和切换}

Moonix中将线程结构定义为如代码 \ref{lst:threadstructure} 的结构体。而实际上,在执行中,并不会用到kstack域,启动和切换基本只涉及contextAddr域。

\begin{minipage}[c]{0.95\textwidth}
\begin{lstlisting}[language={moonix}, caption={线程结构定义}, label={lst:threadstructure}]
typedef struct {
	usize contextAddr;  /* 线程上下文存储的地址 */
	usize kstack;       /* 线程栈底地址 */
} Thread;
\end{lstlisting}
\end{minipage}

类似于中断上下文保存,Moonix也可以将全部的寄存器作为上下文直接保存在栈上。实际实现中,Moonix将线程上下文切换的汇编内联到一个C函数switchContext中,通过调用该函数来进行上下文的切换。根据函数调用规范,在流程进入线程切换的函数时,就会自动保存调用者保存的寄存器到栈上。而在后续在切换回到该线程后,会首先从函数中返回,这部分寄存器的值会自动从栈上被恢复。函数调用规范由编译器负责实现,Moonix无需手动进行保存。采用这种方式进行切换线程上下文,就无需像中断一样保存全部的通用寄存器了。事实上,Moonix只需要手动保存以下的内容:

\begin{itemize}
	\item ra 寄存器,用于保存返回地址
	\item satp 寄存器,保存页表信息,本章的线程都可以说是同属于内核进程的,共用一个页表
	\item s0 \textasciitilde s11,函数调用中被调用者保存的寄存器
\end{itemize}

于是定义线程上下文如代码 \ref{lst:threadcontext}。

\begin{minipage}[c]{0.95\textwidth}
	\begin{lstlisting}[language={moonix}, caption={线程上下文定义}, label={lst:threadcontext}]
		typedef struct {
			usize ra;
			usize satp;
			usize s[12];
		} ThreadContext;
	\end{lstlisting}
\end{minipage}

Moonix在启动后,实际上是位于一个启动线程中的。Moonix需要构造一个“静止”的线程,一个静止线程的结构看起来就像是刚刚被CPU抢占后保存的结构。这样,当其他正在运行的线程切换到它时,就可以将寄存器和栈变成预期的初始化状态,并且跳转到程序的入口开始运行。

Moonix在创建新线程时,会统一将线程的入口ra设置为\_\_restore函数,并创建一个InterruptContext,填充其中的寄存器的值。而线程的真正入口,则保存在InterruptContext中的sepc域中。同时,申请一段内存作为新线程的线程栈,将InterruptContext和ThreadContext分别压入栈底,新线程的线程栈顶,就是ThreadContext结构。

在切换进新线程时,会首先将栈指针切换为新线程的线程栈顶,切换上下文的逻辑会将栈顶的ThreadContext中的内容恢复到CPU中,并执行ret指令跳转到ra寄存器中保存的地址处。这个跳转不会涉及特权模式的变化,跳转之后仍然处于S-Mode。在创建线程时,ra被统一写为\_\_restore函数的地址,此时栈顶收缩至ThreadContext下方的InterruptContext处。

Moonix中断恢复的代码 \ref{lst:restorecontext} 逻辑中,中断恢复时会自动将栈顶的InterruptContext中的内容都恢复到寄存器中,并执行sret跳转到sepc寄存器中保存的地址处。此时InterruptContext恰好位于栈顶。接下来的事情就顺理成章了:中断恢复函数会将InterruptContext中的内容全部恢复到CPU中,包括传入的参数和sepc,随后sret指令将流程跳转到sepc寄存器保存的地址处,即最初设置的线程的真正的入口处。sret指令还可以修改CPU的运行模式,可以通过修改sstatus寄存器的SPP位,来实现不同的运行模式下的线程。

\section{线程池与线程管理}

Moonix中,调度器scheduler只是对线程的id进行调度管理,和实际的线程结构没有直接关系。因此,Moonix使用线程池ThreadPool结构,来给Tid和线程结构建立联系,将调度器对线程Tid的调度转换成线程池对实际线程的调度。线程池的定义如代码 \ref{lst:threadpool} 所示。

通过线程池,线程的调度会被线程池中的Scheduler代理实现,从Scheduler中获取到线程id后,再由线程池在threads数组中操作真正的线程结构。当一个新线程被创建时,也需要由线程池分配一个可用的线程id。

\begin{minipage}[c]{0.95\textwidth}
\begin{lstlisting}[language={moonix}, caption={线程池定义}, label={lst:threadpool}]
/* 线程池中的线程信息槽 */
typedef struct {
	Status status;
	int tid;
	int occupied;       /* 该槽位是否被占用 */
	Thread thread;
} ThreadInfo;
	
typedef struct {
	ThreadInfo threads[MAX_THREAD];
	Scheduler scheduler;
} ThreadPool;
\end{lstlisting}
\end{minipage}

\section{线程调度}

Moonix线程调度依赖于时钟中断,时钟中断的中断处理函数除去预约下一次时钟中断外,还会额外去进行线程调度的工作。

Moonix所有的运行流程都是运行在线程中的,由于要对所有的线程进行调度,Moonix创建了一个调度线程,专用于线程的切换与调度。Moonix中,这个线程被命名为Idle线程。具体来说,Idle线程的作用是:

\begin{enumerate}
	\item 当没有线程在运行时,调度线程根据一定的策略来选择一个线程来执行
	\item 当一个线程被调度器判断需要让出 CPU 控制权时,例如运行时间过长或者运行结束,并不是直接切换到另一个线程,而是先切换到这个调度线程,让调度线程根据一定的策略来选择另一个线程执行
\end{enumerate}

Moonix中定义了一个结构体Processor,用来保存调度线程参与调度所需要的所有信息,如代码 \ref{lst:processor}。其中idle字段就是调度线程,current字段描述了当前正在运行线程的信息,occupied字段表示当前是否有线程(除了调度线程)正在运行。Idle线程所有的调度都是直接依赖Processor结构进行的。

\begin{minipage}[c]{0.95\textwidth}
\begin{lstlisting}[language={moonix}, caption={Processor结构体}, label={lst:processor}]
typedef struct {
	ThreadPool pool;
	Thread idle;
	RunningThread current;
	int occupied;
} Processor;
\end{lstlisting}
\end{minipage}

Idle线程的入口点即为idleMain函数,如代码 \ref{lst:idlemain} 所示。

\begin{minipage}[c]{0.95\textwidth}
\begin{lstlisting}[language={moonix}, caption={idleMain函数}, label={lst:idlemain}]
void
idleMain()
{
	// 进入 idle 时禁用异步中断
	disable_and_store();
	while(1) {
		RunningThread rt = acquireFromPool(&CPU.pool);
		if(rt.tid != -1) {
			// 有线程可以运行
			CPU.current = rt;
			CPU.occupied = 1;
			printf("\n>>>> 从idleMain中切换到线程 %d!\n", CPU.current.tid);
			switchThread(&CPU.idle, &CPU.current.thread);
			
			// 切换回 idle 线程处
			printf("<<<< 从idleMain中切换回idle线程!\n");
			CPU.occupied = 0;
			retrieveToPool(&CPU.pool, CPU.current);
		} else {
			enable_and_wfi();
			disable_and_store();
		}
	}
}
\end{lstlisting}
\end{minipage}

idleMain函数首先关闭中断响应,以防止调度过程被中断打断,造成混乱。接着尝试从线程池中获取一个可执行的线程,切换到该线程开始执行。当某个时刻时间片用完后,内核会切换回idle线程,流程会返回idleMain函数的第16行处,并进入下一次循环。如果没有线程可供执行,idle线程会打开中断。

从一个正在运行的线程的角度来说,整个流程时间线如图 \ref{pic:threadswitch} 所示。

\begin{figure}[htpb]
	\centering
	\includegraphics[width=0.75\textwidth]{threadswitch.png}
	\setlength{\abovecaptionskip}{2pt}
	\caption{线程切换时间线}
	\label{pic:threadswitch}
\end{figure}

在线程用尽时间片,返回到idle线程后,会将上一个线程的资源回收,并防止到线程队列的队尾。随后,idle线程会决策下一个要运行的线程,决策的方式,就是Round-Robin算法。

Round-Robin算法是一种抢占式调度算法,这意味着如果线程在其拥有的时间片内没有完成任务的话,会被内核打断并让出CPU资源,供下一个线程使用。Round-Robin算法会将时间片平等地分配给所有线程,在较长时间内,每个线程都有均等的机会占用CPU。其大致的思想可大致概括为:

\begin{enumerate}
	\item 将所有的线程存入一个优先队列
	\item 每次调度时都取队首的线程占用CPU
	\item 内核定时去检查正在运行的线程是否用完了时间片,如果没有用完就继续执行,否则就将这个线程放到队尾,执行下一个队首线程。
\end{enumerate}

Moonix使用一个双向环形链表来实现队列,链表的节点按照tid+1作为下标存放在数组中,其中下标0处为Dummy Head,用于快速找到队列头。队列中的元素如代码 \ref{lst:rrinfo} 所示,并不保存实际的线程结构,仅仅以下标记录线程的tid。这种方式侵入性较小,调度器只会向内核返回下一个需要执行的线程的tid,耦合度较低,方便替换和扩展不同的调度算法。

\begin{minipage}[c]{0.95\textwidth}
\begin{lstlisting}[language={moonix}, caption={线程调度器元素}, label={lst:rrinfo}]
typedef struct
{
	int valid;
	usize time;
	int prev;
	int next;
} RRInfo;
\end{lstlisting}
\end{minipage}

\section{模块测试}

线程管理模块主要测试线程在运行时间片耗尽后,能否被内核调度抢占,切换到其他线程,并在某个时刻再切换回来。

测试脚本会创建三个用户线程,每个线程都执行同样的功能,就是将自己收到的参数打印1000遍后退出,目的是尽可能得占用时间片,以测试能否成功强占。为了简易起见,给每个线程编的序号就是传入的参数,分别为0-2。将五个线程加载到CPU上执行后的输出代码 \ref{lst:threadtest} 所示,由于输出次数较多,这里简化表示。

\begin{minipage}[c]{0.95\textwidth}
\begin{lstlisting}[caption={线程调度测试}, label={lst:threadtest}]
>>>> 从idleMain中切换到线程 0!
线程 0 开始
00000000000000000000000000000000
<<<< 从idleMain中切换回idle线程!

>>>> 从idleMain中切换到线程 1!
线程 1 开始
11111111111111111111111111111111111111111111111111111111111
线程 1 结束
<<<< 从idleMain中切换回idle线程!

>>>> 从idleMain中切换到线程 2!
线程 2 开始
22222222222222222222222222
<<<< 从idleMain中切换回idle线程!

>>>> 从idleMain中切换到线程 0!
0000000000000000000000000000000
线程 0 结束
<<<< 从idleMain中切换回idle线程!

>>>> 从idleMain中切换到线程 2!
2222222222222222222222222222222222
线程 2 结束
<<<< 从idleMain中切换回idle线程!

暂无可执行线程!
\end{lstlisting}
\end{minipage}

从测试输出中可以看出,在内核开始调度线程时,首先线程0占用了CPU开始执行,在一个时间片之后,线程0没有执行结束,就被idle线程强占,内核重新进行调度,同时线程0被移动到了就绪队列的最后。idle调度线程1占用CPU运行,在线程1占用的一个时间片内,线程1执行完成并正常退出,于是内核重新进行调度,线程2开始占用CPU运行,线程2同样在一个时间片内没有执行完成就被调度线程强占。此时,就绪队列中只剩下了线程0和线程2,在下一轮的调度中它们都成功执行完流程并退出,调度器随后就进入了无就绪线程可调度的状态了。