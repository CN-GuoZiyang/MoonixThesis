% !Mode:: "TeX:UTF-8"

\chapter{shell的实现}[shell]
\label{chapter:shell}

\section{模块概述}

Moonix操作系统在初始化阶段,做的最后一件事,就是加载文件系统的 /bin/sh可执行文件,并将其作为用户进程运行起来。

/bin/sh是一个执行在用户态的shell,是用户和Moonix交互的唯一途径。用户可以直接在shell中键入文件系统上可执行文件的路径,操作系统会将这个文件加载到内核并解析映射,并执行这个可执行文件,在该文件执行结束后,会再切换会shell进程,等待用户的其他命令。除了执行可执行文件外,shell还内建了几个简单的命令,如cd、ls、pwd等,方便用户可以穿梭在文件夹中,方便查看与执行文件。

\section{用户程序的加载与运行}

本节以shell程序为例。

在线程初始化的最后部分,Moonix操作系统从文件系统中加载 /bin/sh,并将其初始化为用户进程,添加到进程调度中。如代码 \ref{lst:initshell} 所示。

\begin{minipage}[c]{0.95\textwidth}
\begin{lstlisting}[language={moonix}, caption={Moonix加载shell}, label={lst:initshell}]
/* 启动终端 */
Inode *shInode = lookup(0, "/bin/sh");
char *buf = kalloc(shInode->size);
readall(shInode, buf);
Thread t = newUserThread(buf);
kfree(buf);
addToCPU(t);
\end{lstlisting}
\end{minipage}

Moonix首先使用文件系统的lookup接口,从文件系统中获取到了保存 /bin/sh文件信息的Inode,并根据Inode中记录的文件大小,使用动态内存分配,在堆空间中开辟了一块缓存buf。随后使用文件系统的readall接口,从文件系统中读取了程序的数据文件,并保存到了buf中。随后根据这个文件构建了一个用户进程,添加到CPU的进程调度中。

构建用户进程的关键点在于解析一个完整的程序文件,这个程序文件是ELF格式的可执行文件。Moonix首先根据ELF头的信息,找到程序头表相对于ELF文件开头的偏移,进而找到文件头表。文件头中的p\_type描述了段类型,Moonix只需要将LOAD类型的段装载进内存即可。内核将代码或数据复制进物理内存之后,需要在页表中映射到虚拟地址空间,页表项的权限需要根据程序头中的p\_flags段权限转化。ELF头的entry字段记录了程序入口的虚拟地址。

Moonix构建用户进程时,除去要映射用户程序的代码和数据到虚拟地址空间,还需要将内核映射到虚拟地址空间中,以保证用户进程在发起环境调用时,进程进入S-Mode,能够顺利进入内核处理调用。于是在构建用户进程时,解析ELF文件之前,Moonix首先会填充一个已经映射好内核的页表,随后在这个页表的基础上继续映射程序私有的代码数据。

\section{用户态环境调用}

shell是一个U-Mode进程,而内核运行在S-Mode下。通常,一个U-Mode进程的能力十分有限,大部分能力都需要依赖操作系统功能,甚至输出一个字符到屏幕上,都必须通过操作系统。U-Mode环境与内核唯一的沟通渠道,就是通过环境调用的方式。

RISC-V规范规定,环境调用的调用号通过a7寄存器传递,参数通过a1、a2和a3寄存器,处理结果的返回值通过a0寄存器传递。如代码 \ref{lst:shell} 所示,Moonix通过内联汇编的方式,执行ecall指令,并在执行指令之前填充参数和调用号寄存器。

当一个U-Mode环境调用发生时,Moonix内核会接收到一个系统中断,中断类型是USER\_ENV\_CALL。Moonix随后会从a7寄存器中取出中断调用号,从a1、a2、a3寄存器中取出参数,根据调用号的类型进行不同的处理,并将返回值写入a0寄存器中,结束这次中断处理。

当shell通过EXEC环境调用执行某个可执行文件时,由于此时shell也在运行状态,由于被执行的进程和shell处于并发执行状态,两个进程的输出可能会混杂到一起。所以在执行EXEC系统调用时,Moonix会挂起发起系统调用的线程,使得shell进入休眠状态,直到被执行的进程结束,shell进程才会被重新调度。

\begin{minipage}[c]{0.95\textwidth}
\begin{lstlisting}[language={moonix}, caption={U-Mode环境调用}, label={lst:shell}]
typedef enum {
	Shutdown = 13,
	LsDir = 20,
	CdDir = 21,
	Pwd = 22,
	Open = 56,
	Close = 57,
	Read = 63,
	Write = 64,
	Exit = 93,
	Exec = 221,
} SyscallId;

#define sys_call(__num, __a0, __a1, __a2, __a3)                          \
({                                                                  \
	register unsigned long a0 asm("a0") = (unsigned long)(__a0);    \
	register unsigned long a1 asm("a1") = (unsigned long)(__a1);    \
	register unsigned long a2 asm("a2") = (unsigned long)(__a2);    \
	register unsigned long a3 asm("a3") = (unsigned long)(__a3);    \
	register unsigned long a7 asm("a7") = (unsigned long)(__num);   \
	asm volatile("ecall"                                            \
	: "+r"(a0)                                          \
	: "r"(a1), "r"(a2), "r"(a3), "r"(a7)                         \
	: "memory");                                        \
	a0;                                                             \
})
\end{lstlisting}
\end{minipage}