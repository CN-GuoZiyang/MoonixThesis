% !Mode:: "TeX:UTF-8"

\chapter{中断处理模块设计与实现}[interrupt]
\label{chapter:interrupt}

\section{模块设计}

\ref{sec:interrupt} 节中介绍过,RISC-V支持两种方式的中断处理,分别是Direct模式与Vectored模式。采用Direct模式时,当任意中断发生后,都会跳转到stvec中的BASE地址处;若是采用Vectored模式,则类似于Linux内核中使用的中断向量表,需要首先进行向量表的填充,不利于统一处理或忽略一些中断,且填表过程比较繁琐,所以Moonix中采用Direct模式来对中断进行统一处理。在中断处理函数中就可以很方便地根据中断类型来对不同的类型调用不同的处理函数分流处理流程。

在跳转进入中断处理函数之前,需要保存当前线程的CPU状态,并在退出之前恢复,以保证整个中断处理过程对线程来说是透明的。Moonix将CPU中所有的通用寄存器都保存在线程的内核栈上,并在中断函数处理结束之后将寄存器的值从内核栈上恢复到CPU,随后再跳转回原线程被中断打断的位置继续执行。

在Moonix中,中断处理需要处理以下三种最重要的中断:时钟中断、用户态(U-Mode)环境调用和外部串口中断。时钟中断主要用于线程调度,在每次时钟中断发生时,线程调度模块都会检查当前线程的时间片,来决定是否要继续执行,更具体内容将在第 \ref{chapter:thread} 章展开说明。用户态环境调用类似于Linux中的系统调用,用于U-Mode下进程向S-Mode下操作系统请求系统功能,如输入输出、文件读写等。外部串口中断主要用于实现用户输入,因为在QEMU中,键盘被实现为一个串口设备。用户输入是一个异步事件,不可预知,为了避免需要读取键盘的线程陷入无意义的忙循环等待中,Moonix使用条件变量来实现条件等待-恢复机制,条件变量与标准输入的实现细节将在 \ref{sec:condition} 节做更深入的讨论。

\section{中断上下文的保存与恢复}

\section{条件变量与标准输入的实现}
\label{sec:condition}